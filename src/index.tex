
\chapter{Periodicity, Reciprocal Lattice, Floquet Modes}
\label{chap:periodicity}

This chapter discusses the direct and reciprocal lattice vectors, and
defines the Floquet modes that can exist in the dielectric regions.

\section{The Direct Lattice}
We consider a structure with discrete translational invariance in two space
dimensions.  The periodicity is characterized by the {\em direct lattice
  vectors} $\s_1$ and $\s_2$, a pair of real vectors satisfying
\begin{equation}
  \s_1 \bdot \z = \s_2 \bdot \z = 0, \quad A \equiv \z \bdot \s_1 \cross \s_2 > 0.
\end{equation}
The structure is invariant to a translation consisting of any integer number of
shifts in the $\s_1$ or $\s_2$ directions.  Such periodicity
is exhibited by idealized models of frequency selective surfaces
(FSSs) and phased arrays, for example.
This periodicity gives rise to the concept of the {direct lattice},
the set of points $\vecrho_{mn} = \x x_{mn} + \y y_{mn}$ satisfying
\begin{equation}
  \vecrho_{mn} = m \s_1 + n \s_2, \quad \text{for $m$ and $n$ any integers.}
\end{equation}
A periodic structure and its direct lattice is shown in
Figure~\ref{fig:direct}.
\begin{figure}[htbp]
  \begin{center}
        \fbox{%
      \psset{unit=0.005in}
      \pspicture*(-1,-10)(660,410)
                                % Paint it on a white, opaque background:
      \psframe*[linecolor=white,fillcolor=white,fillstyle=solid](-1,-10)(660,410)
      \multips(-60.62,-105.0)(60.62,105.0){5}%
      {%
        \multips(-242.48,0)(121.24,0){8}%
        {%
          \pspolygon(73.68,18.0)(108.18,18)(130.77,52.5)%
          (108.18,87.0)(73.68,87.0)(51.096,52.5)%
          %
          \psline[linestyle=dashed,linewidth=0.5pt](0,0)(121.24,0)
          \psline[linestyle=dashed,linewidth=0.5pt](0,0)(60.62,105.0)
          \qdisk(0,0){1.5pt}
        }%
      }
    \rput(181.86,105){%
      \psline[linewidth=1pt]{->}(0,0)(121.24,0)
      \psline[linewidth=1pt]{->}(0,0)(60.62,105.0)
      \rput(90,-13){$\s_1$}
      \rput[l](10,90){$\s_2$}
      }
    \endpspicture
      }
    \caption{A frequency selective surface consisting of a thin metal plate
        with hexagonal perforations, and the associated direct
        lattice. The location selected for the lattice origin is arbitrary.}
    \label{fig:direct}
  \end{center}
\end{figure}

\section{Periodic Boundary Conditions and the Unit Cell}
\label{sec:pbcuc}
We now assume that an electromagnetic excitation of some type is
applied to the structure.  In the case of a FSS, the excitation takes
the form of an incident plane wave.  In the case of a phased array,
the excitation may be an incident plane wave, or perhaps a set of
incoming waveguide modes in each of the excitation ports of the
radiating elements.  Denote the spatial variation of the excitation by
the function $V(\r)$.  We insist that the function $V$ satisfy the
following quasi-periodicity condition:
\begin{equation}
  \label{eq:floquetbc}
  V(\r + m\s_1 + n\s_2) = V(\r) e^{-j(m\psi_1 + n\psi_2)}, \quad
  \text{for any integers $m$ and $n$}
\end{equation}
where $\psi_1$ and $\psi_2$ are given real numbers, which we will
refer to as the ``unit cell incremental phase shifts''.  By the
translational invariance of Maxwell's equations and given the discrete
translational invariance of the structure, it is clear that all
electromagnetic fields, charges, etc., resulting from the given
excitation must also satisfy \eqref{eq:floquetbc}, which we refer to
as the ``Floquet boundary condition.''

Since the fields throughout the structure satisfy
Equation~\eqref{eq:floquetbc}, it suffices to restrict consideration
to a single unit cell $U$, defined\footnote{The definition of a unit
  cell is not unique.  The present definition is most useful for our
  purposes.}
as the set of points $\r$
satisfying
\begin{equation}
  \label{eq:unitcell}
  U = \{\r: \; \r = \xi_1 \s_1 + \xi_2 \s_2 + \z z, \quad 0 \leq \xi_1 , \xi_2
  \leq 1\},
\end{equation}
where $\xi_1$ and $\xi_2$ are the so-called ``normalized area
coordinates,'' each constrained to the interval $[0,1]$.
We seek a set of modes that can propagate in the unit cell, subject to
an appropriate set of boundary conditions to be stated below.  Let
$E(\r)$ be some rectangular component of electric or magnetic field
evaluated at a point $\r = \x x + \y y + \z z = \xi_1 \s_1 + \xi_2
\s_2 + \z z$ within the unit cell.  
%Let $f(\xi_1,\xi_2) = 
%E(\xi_1\s_1+\xi_2\s_2+\z z) =E(\r)$.
Then the quasi-periodic boundary condition can be expressed as
\begin{subequations}
  \label{eq:cellfloquetbc}
  \begin{align}
    E(\s_1 + \xi_2 \s_2 + \z z) &= E(\xi_2 \s_2 + \z z) e^{-j\psi_1} \\
    E(\xi_1 \s_1 + \s_2 + \z z) &= E(\xi_1 \s_1 + \z z) e^{-j\psi_2} 
  \end{align}
\end{subequations}
  which must hold for all $z$ and for all $\xi_1$ and $\xi_2$ in the
  interval $[0,1]$.  


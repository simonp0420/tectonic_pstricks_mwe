This chapter discusses the direct and reciprocal lattice vectors, and
defines the Floquet modes that can exist in the dielectric regions.

\section{The Direct Lattice}
We consider a structure with discrete translational invariance in two space
dimensions.   A periodic structure and its direct lattice is shown in
Figure~\ref{fig:direct}.
\begin{figure}[htbp]
  \begin{center}
        \fbox{%
      \psset{unit=0.005in}
      \pspicture*(-1,-10)(660,410)
                                % Paint it on a white, opaque background:
      \psframe*[linecolor=white,fillcolor=white,fillstyle=solid](-1,-10)(660,410)
      \multips(-60.62,-105.0)(60.62,105.0){5}%
      {%
        \multips(-242.48,0)(121.24,0){8}%
        {%
          \pspolygon(73.68,18.0)(108.18,18)(130.77,52.5)%
          (108.18,87.0)(73.68,87.0)(51.096,52.5)%
          %
          \psline[linestyle=dashed,linewidth=0.5pt](0,0)(121.24,0)
          \psline[linestyle=dashed,linewidth=0.5pt](0,0)(60.62,105.0)
          \qdisk(0,0){1.5pt}
        }%
      }
    \rput(181.86,105){%
      \psline[linewidth=1pt]{->}(0,0)(121.24,0)
      \psline[linewidth=1pt]{->}(0,0)(60.62,105.0)
      \rput(90,-13){$s_1$}
      \rput[l](10,90){$s_2$}
      }
    \endpspicture
      }
    \caption{A frequency selective surface consisting of a thin metal plate
        with hexagonal perforations, and the associated direct
        lattice. The location selected for the lattice origin is arbitrary.}
    \label{fig:direct}
  \end{center}
\end{figure}

